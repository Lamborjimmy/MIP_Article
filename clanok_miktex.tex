
\documentclass[10pt,twocolumn,twoside,slovak,a4paper]{article}
\usepackage{graphicx} % Required for inserting images
\usepackage{floatflt}
\usepackage{graphicx}
\title{\textbf{Examples of search algorithms used in information retrieval}}
\author{Jaroslav Ertl}
\date{October 2023}

\begin{document}

\maketitle
 
\begin{abstract}
    Many people use search engines on a daily basis for information retrieval(IR), whether it's for school, work or personal use, but not many people know how the logic behind it works. One of the key factors to achieve the desired result you're looking for are algorithms. In IR, an algorithm can be defined as a set of complex mathematical and computation techniques to retrieve relevant information from a large dataset. There are many types of search algorithms used in IR, each of them having its own unique purpose. 

\begin{floatingfigure}[r]{0.3\textwidth}
    \centering
  \includegraphics[width=0.18\textwidth]{StuFiit.png} % replace with your actual file name
  \caption{FIIT STU}
\end{floatingfigure}
  You can have a graph-based algorithm, keyword-based algorithm and many more. In this article, we will dive into the world of different search algorithms and their application in IR. In each section, we will discuss different search algorithm, how they work and their purpose.
    
    Adding for github purpose only.
 \begin{figure}
 \centering
\includegraphics[width=0.5\textwidth]{UmletDiagram.png} % replace with your actual file name
\caption{Test Diagram}
\end{figure}
You can find the article that guided me towards the topic of search algorithms in references.\cite{10210566}
\end{abstract}

\bibliography{literatura}
\bibliographystyle{abbrv} 
\cite{Chonyy}
\end{document}
